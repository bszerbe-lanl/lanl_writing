\documentclass{article}

% Language setting
% Replace `english' with e.g. `spanish' to change the document language
\usepackage[english]{babel}

% Set page size and margins
% Replace `letterpaper' with `a4paper' for UK/EU standard size
\usepackage[letterpaper,top=2cm,bottom=2cm,left=3cm,right=3cm,marginparwidth=1.75cm]{geometry}

% Useful packages
\usepackage{afterpage}
\usepackage{amsmath}
\usepackage{graphicx}
\usepackage[colorlinks=true, allcolors=blue]{hyperref}
\usepackage{subcaption}
\newcommand{\dead}{\tau_{\text{dead}}}
\newcommand{\overlap}{t_{\text{over}}}

\title{Informatic effects of dead-time measurements on the determination of the incoming rate of events}
\author{Brandon Zerbe}

\begin{document}
\maketitle

\section{Overview}

For many detectors, an important limitation, especially for
large signals, is
known as dead-time.
Dead-time is a period of time after an obervation during which
additional detection is not possible.  Such
limitations are fundamental to most detectors as can be
seen in both electronic and biological systems; namely,
electronic detectors employing scitillator detectors as well as biological
systems that employ visual detectors (eyes)
initiate down-stream signals interpretted as a detected 
event from ``chemical'' excitation
of detector-specific materials,
and the materials used to initiate this signal
have a characteristic relaxation time before they can again
be chemically excited.  It is during this time that the detector is said to be dead as
any incoming event arriving while the material is relaxing 
will not excite the material and therefore will
not initiate the down-stream signal, i.e.\ the event will not be detected.

The effect of this process on the statistics seen by the detector have
long been treated within the literature[ref]; however, we have a couple
concerns about this work.
The first concern is theoretical; the authors use the framework of
a "renewal process" that is not standard within the theoretical physics community.
The second concern is practical; the analytic equations for describing the
distribution are derived by the authors are expressed 
in terms of differences
between CDF functions.
This is a concern as these equations are therefore
written in terms of differences of large sums of incomplete gamma functions
that are difficult to work with both computationionally and analytically. 

Fortunately, we are aware
of an alternative derivation of the
analytic equations for the distribution using
theoretically standard Poisson processes.
In this document, we discuss first the derivation
of the Poisson distribution from a Poisson process
and then generalize this to the case when dead-time effects
are considered.

\section{Stochastic modelling}\label{sec:Stochastics}

The ultimate goal of the stochastic (forward) modelling in this document 
is to derive the probability distribution for the number
of counts, $n$, seen during the observation time, $T$, 
given some rate, $\rho$, of incoming events.
Without dead-time effects, this distribution is just the 
Poisson distribution, which we denote as $p_{\rho, T}(n)$;
in the presence of a dead-time, with a dead-time duration of $\dead$,
we call the distribution the ``dead-time modified Poisson distribution''
denoted by $p_{\rho, T, \dead}(n)$.

Before proceeding to the derivation of the dead-time modified Poisson 
distribution, we first 
present some mathematical background of the features
of Poisson processes that allow us to derive the Poisson Distribution.
Namly, we assume that the reader is not intimately familiar with Poisson processes
and therefore requires such background.  Feel free to skip the next
section and continue onto the derivation of the dead-time modified
Poisson distribution if you are not such a reader.

\subsection{Poisson processes}

Process models typically assume the random arrival of events, henceforth 
the singular of which will be called ``arrival'', along some continuous 
variable, which for our problem is time.   These models usually assume that
the randomness is not affected by the time considered; in other words, resetting
the ``0'' time does not change the model, and the rate at which these arrivals take
place can be characterized by a constant ``rate parameter".
Finally, the randomness is then modelled by
choosing some probability distribution for the next arrival whose mean reproduces the
rate parameter.

The most commonly used process model utilizes the exponential distribution for the
randomness; such a model is called a ``Poisson process''.  Within this model,
the probability that the next arrival occurs between times $t$ and $t + dt$, 
$p(t)$, is given by
\begin{subequations}
  \begin{align}
	  p(t) &=  e^{-\rho t} \rho dt \label{eq:exp-dist}
  \end{align}
Often, this probability is equivalently 
characterized by the inter-arrival time density,
  \begin{align}
  	f(t) &= \rho e^{-\rho t} \label{eq:exp-den}
  \end{align}
so that $p(t) = f(t) dt$ although this notation will not
be used further here and is only mentioned to be consistent
with existent literature.
Often it is more convenient to look at the cumulative
distribution function (CDF) for the probability;
for this probability, the CDF is defined as
	\begin{align}
		P(T) &= \int_0^T p(t) ~ dt\nonumber\\
				 &= 1 - e^{-\rho t} \label{eq:exp-cdf}
	\end{align}
\end{subequations}
and represents the total probability that the next
event will occur within time $t$.  We will use
the CDF later within this document.

It is from this fundamental process model that both the Poisson and the 
related Erlang distributions can be derived.  Next,
we will derive just the
the Poisson distribution as this derivation will provide
a sufficient framework to modify with the dead-time effect
later.

\subsubsection{Poisson distribution}

The question we ask is ``Assuming a Poisson process,
what is the distribution of the number of
observed events, $n$, if the observation duration is $T$
and the rate parameter is $\rho$?``
The answer to this question is the Poisson distribution and will
be derived here.

We first determine the probability that $0$ events are observed,
which we denote as $p_{T}(0 | \rho)$.  Note however, that we can restate
this probability as the total probability that the next event
does not arrive within the observation period $T$, which is simply the
complement of Eq. (\ref{eq:exp-cdf}), i.e.
\begin{align}
	p_{\rho, T}(0) &= 1 - P(T) &= e^{-\rho T} 
\end{align}

Next, we can write the probability that $1$ particle is observed, which
we denote as $p_{\rho, T}(1)$.  To derive this, let us examine the situation
when a single event arrives between times $t$ and $t + dt$
and no other event arrives afterwards.
The probability of this occuring, $a_1(t)$, is the product of the probability
that the next arrival occurs at time $t$, $p(t)$, and the probability
that no particles aftwards, $p_{\rho, T - t}(0)$:
\begin{align}
	a_1(t) &= p(t) p_{\rho, T - t}(0)\nonumber\\
			 &= \rho e^{-\rho T}dt\nonumber
\end{align}
Of course, a particle may arrive at any time $t$, so 
we need to sum up all probabilities associated
with the possible
arrival times to determine $p_{\rho, T}(n = 0)$:
\begin{align}
	p_{\rho, T}(1) &= \int_0^T a_1(t)\nonumber\\
	               &= \int_0^T \rho e^{-\rho T} dt\nonumber\\
								 &= \rho T e^{-\rho T}
\end{align}

It should be apparent that the expression $a_1(t)$ suggests a recursion relation; specifically,
we may generalize $a_n(t) = p(t) p_{\rho, T - t}(n - 1)$.  Such recursion relations
suggest a derivation by induction which requires a ``guess'' of the form of 
$p_{\rho, T}(n - 1)$ that agrees with prior expressions 
and a proof that $p_{\rho, T}(n)$ satisfies that ``guess'' as well.
Typically, this
guess is intuited by explicitly deriving several more $p_{\rho, T}(n)$ for additional
$n$'s; we will need to do this for the dead-time modified Poisson distribution.
However, we forgo this intuition as the form of the Poisson distribution is already 
well known.  Specifically, the guess for Poisson will be
\begin{subequations}\label{eq:Poisson-dist}
\begin{align}
	p_{\rho, T}(n-1) = \frac{{(\rho T)}^{n-1}}{(n -1)!} e^{-\rho T}\label{eq:Poisson-guess}
\end{align}
and we will show that $p_{\rho, T}(n)$ likewise follows this form finishing our ``proof'' by
induction.

	Namely, introduce $a_n(t) = p(t) p_{\rho, T - t}(n - 1)$ represents, which
the probability that the next event occurs between times $t$ and $t + dt$ and that
$n-1$ events occur in the remaining time.  Defining the substitution $u = T- t$, 
the probability that $n$ events occur during
$T$ is then given by
  \begin{align}
	  p_{\rho, T}(n) &= \int_0^T a_n(T-t)\nonumber\\
									 &= \int_0^T \rho e^{-\rho t} \frac{{(\rho (T - t))}^{n-1}}{(n -1)!} e^{-\rho (T-t)} dt \nonumber\\
									 &= \int_0^{\rho T} e^{u -\rho T} \frac{{u}^{n-1}}{(n -1)!} e^{-u} du \nonumber\\
									 &= e^{-\rho T} \int_0^{\rho T} \frac{{u}^{n-1}}{(n -1)!} du \nonumber\\
									 &= \frac{{(\rho T)}^n}{n!} e^{-\rho T}\label{eq:Poisson-verify}
  \end{align}
We see that this has the same form as Eq. (\ref{eq:Poisson-guess}), so our ``proof'' is done.
\end{subequations}

\subsection{Dead-time modified Poisson distribution}

We now modify the derivation of the Poisson distribution in order to obtain an analytic form for the
dead-time modified distribution.  
We denote the dead-time by the parameter $\dead$ and the probability of seeing $n$ events
that have an incoming rate of $\rho$ during an observation period of $T$ by
$p_{\rho, T, \dead}(n)$.
In short, dead-time has one main effect on the analysis; 
dead-time reduces the amount of time the Poisson process has available during
which we may``observe'' events.

This effect can be separated into two quantitative pieces:
\begin{enumerate}
	{\item{The observation of a particle results in subsequent events having $\dead$ less
		time during which events can be observed.}}
	{\item{The overall period of observation is reduced as there may be some overlap
		of a dead-time period that extends from the previous observation period.}}
\end{enumerate}
Denote the second, currently unknown time as $\overlap$, and note that it is itself
and random variable.  Likewise, in some detectors $\dead$ may be a random variable,
but in this treatment we are assuming a non-paralyzable dead-time where $\dead$
is treated as a constant.

We first derive the dead-time modified Poisson distribution
assuming $\overlap$ is known but is dependent of both $\dead$ and $\rho$;
after obtaining the expression in terms of $\overlap$, $\dead$, $\rho$ and $T$,
we then incorporate the random-nature of the overlap time and it's dependence
on the dead-time and rate parameter to obtain
an expression fully in terms of $\dead$, $\rho$ and $T$.

We begin again with the probability of $n = 0$, $p_{T, \dead}(0 | \rho)$.  Note that since we have 
some unknown, $\overlap$, during our observation window,
the amount of time available for the Poisson process to occur is given the observation
window less this overlap, $T - \overlap$.  By the argument we made to obtain the 
Poisson distribution, we can write this probability as
\begin{subequations}
\begin{align}
	p_{\rho, T, \dead}(0) &= 1 - P(T - \overlap) 
\end{align}
The above derivation assumes $T - \overlap > 0$; however, if it isn't,
the the entire observation period
is blocked by the dead-time effect.
In other words, we know that in such a situation, we can't observe any.
Making this explicit gives the full expression for
this probability of
\begin{align}
	p_{\rho, T, \dead}(0)
	  &= \begin{cases}
			     e^{-\rho(T - \overlap)} & T  > \overlap\\
			     1                       & \text{else}
			 \end{cases}
\end{align}
\end{subequations}
Note that when $\dead \to 0$, $\overlap \to 0$ and we reproduce the standard
Poisson probability for $0$ arrivals as should be expected.

Next we start to derive the probability of $n = 1$, $p_{\rho, T, \dead}(1)$
by first obtaining an expression for the probability analogous to 
$a_1$ in the derivation of the Poisson distribution that we will call $b_1$.  Note
that after an arrival, the amount of time within the 
observation period remaining will be reduced by $\dead$.  Therefore, we
must break up our probability
an arriving occuring between $t$ and $t + dt$ and no further particles
arriving in the reamining time into two pieces:
\begin{enumerate}
	\item{$0 <= t < T - \overlap - \dead$: the probability that the next arrival
		is at time $t$ and the probability that $0$ arrivals occurr
		during the remaining time, $T - t - \overlap - \dead$.}
	\item{$T - \overlap - \dead <= t <= T - \overlap$: the probability that the next arrival
    is at time $t$.  The dead time associated with enforces
		no further particles arriving and contributes to 
		the overlap with the next observation period.}
\end{enumerate}
Mathematically, this probability looks like
\begin{align}
	b_1(t)
	  &= \begin{cases}
			   p(t) p_{\rho, T - t - \dead, \dead}(0) & t \in [0, T - \overlap - \dead]\\
				 p(t)                             & t \in [T - \overlap - \dead <= t <= T - \overlap]\\
				 0                                & \text{else}
			 \end{cases}\nonumber\\
		&= \begin{cases}
			\rho e^{-\rho (T - \overlap - \dead)}dt & t \in [0, T - \overlap - \dead] \text{ and } T > \overlap + \dead\\
				 \rho e^{-\rho t}dt                      & t \in [T - \overlap - \dead, T - \overlap] \text{ and } T > \overlap + \dead \\
				 \rho e^{-\rho t}dt                      & t \in [0, T - \overlap] \text{ and } T \in [\overlap, \overlap + \dead] \\
				 0                                       & \text{else}
		\end{cases}\label{eq:b_1}
\end{align}

Analogous to the derivation of the Poisson distribution, we integrate over the probability
in Eq. (\ref{eq:b_1})
to obtain the probability of seeing one event during the observation time.  First we assume
$T > \overlap + \dead$:
\begin{subequations}
\begin{align}
	p_{\rho, T, \dead}(1) &= \int_0^{T-\overlap - \dead} b_1(t)\nonumber\\
												 &= \int_0^{T-\overlap - \dead} \rho e^{-\rho (T - \overlap - \dead)}dt + \int_{T - \overlap - \dead}^{T - \overlap} \rho e^{-\rho t}dt\nonumber\\
												 &= \rho (T-\overlap - \dead) e^{-\rho (T - \overlap - \dead)} + e^{-\rho (T - \overlap - \dead)} - e^{-\rho (T - \overlap)}\nonumber\\
												 &= \left( \rho (T-\overlap - \dead) + 1 - e^{-\rho\dead}\right) e^{-\rho (T - \overlap - \dead)}
\end{align}
In the case that
$T \in [\overlap, \overlap + \dead]$, the first integral and the lower limit of the second integral become $0$, and we
need only evalute the second integral for the lower- and upper- limits of $0$ and $T - \overlap$, respectively.
This evaluation gives $1 - e^{-\rho(T - \overlap)}$.
Therefore, the total probability expression can be written as
\begin{align}
	p_{\rho, T, \dead}(1)
	  &= \begin{cases}
			\left( \rho (T-\overlap - \rho \dead) + 1 - e^{-\rho\dead} \right) e^{-\rho (T - \overlap - \dead)} & T > \overlap + \dead\\
			1 - e^{-\rho(T - \overlap)}                                                                   & T \in [\overlap, \overlap + \dead]\\
			0                                                                                             & \text{else}
			 \end{cases}
\end{align}
\end{subequations}
Note that when $\dead \to 0$, we again reproduce the Poisson distribution for $1$ arrival
as should be expected. 

To make subsequent analysis simpler, we now introduce the following notations:
\begin{subequations}
\begin{align}
	c_n &= n \rho \dead\\
	d_{n}(T) &= d_n = \rho(T - \overlap - n \dead) = \rho(T - \overlap) - c_n
\end{align}
\end{subequations}
In this notation, the dead-time modified distribution for $n=0$ and $n=1$ are given by
\begin{align*}
	 p_{\rho, T, \dead}(0)
	    &= \begin{cases}
			     e^{-d_0} & T > \overlap\\
				   0        & \text{else}
			   \end{cases}\\
	  p_{\rho, T, \dead}(1)
	    &= \begin{cases}
				(d_0 + 1 - c_1 - e^{-c_1})e^{-d_1} & T > \overlap + \dead\\
				1 - e^{-d_0}               & T > [\overlap, \overlap + \dead]\\
				   0        & \text{else}
			   \end{cases}\\
	\end{align*}

Next we move onto obtaining an expression for the probability 
of an arrival occuring between $t$ and $t + dt$ and the subsequent
arrival occurring in the remaining time, $b_2$.
Unlike when we had a single arrival above, we need not consider the first arrival occurring
during $[T - \overlap - \dead, T - \overlap]$ as this blocks off the rest of the
observation time from allowing the second arrival.
Note also $d_1(T - t - \dead) = \rho(T - t - 2 \dead) = d_1(T) - \rho \dead - \rho t$
where we will continue to denote $d_1 = d_1(T)$.
Therefore, the probability can be written as
\begin{subequations}
\begin{align}
	b_2(t)
	  &= \begin{cases}
			   p(t) p_{\rho, T - t - \dead}(1) & t \in [0, T - \overlap - \dead]\\
				 0                                & \text{else}
			 \end{cases}\nonumber\\
	  &= \begin{cases}
			\rho e^{-\rho t} \left( d_0 - \rho \dead - \rho t + 1 - c_1 - e^{-c_1}\right)e^{-d_1 + \rho \dead + \rho t}~dt & t \in \left[0, \frac{d_1}{\rho}\right] \text{ and } T - \dead > \overlap + \dead\\
				 \rho e^{-\rho t}(1 - e^{-d_1 + \rho \dead + \rho t})dt                                                               & t \in \left[0, \frac{d_1}{\rho}\right] \\
				  & \text{ and } T - \dead \in [\overlap, \overlap + \dead]\\
			    0                                                                                         & \text{else}
			 \end{cases}\nonumber\\
	  &= \begin{cases}
			\rho \left( d_1 - \rho t + 1 - c_1 - e^{-c_1}\right)e^{-d_2}~dt & t \in \left[0, \frac{d_1}{\rho}\right] \text{ and } T > \overlap + 2 \dead\\
				 \rho (e^{-\rho t} - e^{-d_2})dt                                                               & t \in \left[0, \frac{d_1}{\rho}\right] \\
				  & \text{ and } T \in [\overlap + \dead, \overlap + 2 \dead]\\
			    0                                                                                         & \text{else}
		\end{cases}\label{eq:b_2-pre}
\end{align}
Now we introduce the substitutions
$u = d_1 - \rho t$ and
$v = \rho t$. Now Eq. (\label{eq:b_2-pre}) can be written as
\begin{align}
	b_2(t) 
	  &= \begin{cases}
			(u + 1 - c_1 - e^{-c_1}) e^{-d_2} ~ du & u \in [0, d_{1}] \text{ and } T > \overlap + 2 \dead\\
			 (e^{-v} - e^{-d_2}) ~ dv            & v \in [0, d_1] \text{ and } T \in [\overlap + \dead, \overlap + 2 \dead]\\
			    0                   & \text{else}
		\end{cases}\label{eq:b_2-with-constants}
\end{align}
where the change in the order of the interval is due to the sign on $t$ in the $u$ substitution.
Integrating this gives
%MISTAKE
\begin{align}
	p_{\rho, T, \dead}(2) 
	  &= \begin{cases}
			\frac{1}{2} d_{1}^2 e^{-d_{2}} + (1 - c_1 - e^{-c_1})d_{1} e^{-d_2} & T > \overlap + 2 \dead\\
				 1 - e^{-d_1} - d_1 e^{-d_2}                             & T \in [\overlap + \dead, \overlap + 2 \dead]\\
				 0                                               & \text{else}
			 \end{cases}\nonumber\\
	  &= \begin{cases}
			\left(\frac{1}{2} (d_{0}^2 - 2 c_1d_0 + c_1^2) + (1- c_1 - e^{-c_1})(d_{0} - c_1)\right) e^{-d_2} & T > \overlap + 2 \dead\\
				 1 - e^{-d_1} - d_1 e^{-d_1 + c_1}                             & T \in [\overlap + \dead, \overlap + 2 \dead]\\
				 0                                               & \text{else}
			 \end{cases}\nonumber\\
	  &= \begin{cases}
			\left(\frac{1}{2} d_{0}^2 + (1 - 2 c_1 - e^{-c_1}) d_{0}  - (1 - c_1 - \frac{1}{2} c_1^2 - e^{-c_1})\right)e^{-d_2} & T > \overlap + 2 \dead\\
				 1 - e^{-d_1}(1 - d_1 e^{c_1})                     & T \in [\overlap + \dead, \overlap + 2 \dead]\\
				 0                                               & \text{else}
			 \end{cases}
\end{align}
\end{subequations}
Here, note that $d_n = \rho (T - \overlap - n \dead) = d_0 - c_n$.

We do one last calculation to obtain $p_{T, \dead}(3 | \rho)$, where we first need to
calculate $b_3(t)$, the probability that the first event occurs at $t$ and the remaining
$2$ events occur during the remaining time.  Again, we use
$d_1(T - \dead - t) = d_1 - \rho \dead - \rho t$ and also
$d_2(T - \dead - t) = d_2 - \rho \dead - \rho t$:
\begin{subequations}
\begin{align}
	b_3(t)
	  &= \begin{cases}
			p(t) p_{\rho, T - t - \dead, \dead}(2) & t \in [0, T - \overlap - \dead]\\
				 0                                & \text{else}
			 \end{cases}\nonumber\\
	  &= \begin{cases}
			\rho e^{-\rho t} \bigg(\frac{1}{2} (d_{0} - \rho \dead - \rho t)^2 + (1 - 2 c_1 - e^{-c_1}) (d_{0} - \rho \dead - \rho t) & T - \dead > \overlap + 2 \dead\\
			 \quad \quad - (1 - c_1 - \frac{1}{2} c_1^2 - e^{-c_1}) \bigg)e^{-d_2 + \rho \dead + \rho t} dt& \text{and } t \in \left[0, T - \overlap - \dead \right]\\
				 \rho e^{-\rho t}(1 - e^{-d_1 + \rho \dead + \rho t}(1 - (d_1 -\rho \dead - \rho t ) e^{c_1})) & T \in [\overlap + 2 \dead, \overlap + 3 \dead]\\
				 & \text{and } t \in \left[0, T - \overlap - \dead \right]\\
				 0                                               & \text{else}
		\end{cases}\nonumber\\
	  &= \begin{cases}
			\rho \bigg(\frac{1}{2} (d_{1} - \rho t)^2 + (1 - 2 c_1 - e^{-c_1}) (d_{1} - \rho t) & T - \dead > \overlap + 2 \dead\\
			 \quad \quad - (1 - c_1 - \frac{1}{2} c_1^2 - e^{-c_1}) \bigg)e^{-d_3} dt& \text{and } t \in \left[0, T - \overlap - \dead \right]\\
				 \rho (e^{-\rho t} - e^{-d_2}(1 - (d_2 - \rho t ) e^{c_1})) & T \in [\overlap + 2 \dead, \overlap + 3 \dead]\\
				 & \text{and } t \in \left[0, T - \overlap - \dead \right]\\
				 0                                               & \text{else}
		\end{cases}\label{eq:b_3-pre}
\end{align}
Now sub $u_1 = d_1 - \rho t$ and $u_2 = d_2 - \rho t$ and $v = \rho t$, we get
\begin{align}
	b_3(t)
	  &= \begin{cases}
			\left(\frac{1}{2} u_1^2 + (1 - 2 c_1 - e^{-c_1})u_1 - (1 - c_1 - \frac{1}{2} c_1^2 - e^{-c_1})\right)du_1 ~ e^{-d_3} & T > \overlap + 3 \dead\\
			 & \text{and } u_1 \in [0, d_1]\\
				 \left(e^{-v} - e^{-d_2}\right) dv + u_2 e^{-d_3} ~ d u & T \in [\overlap + 2 \dead, \overlap + 3 \dead]\\
			 & \text{and } v \in [0, d_1] \text{and } u_2 \in [-c_1, d_2]\\
				 0                                               & \text{else}
		\end{cases}\label{eq:b_3-pre}
\end{align}
and doing the integration gives
\begin{align}
	p_{\rho, T, \dead}(3) 
	  &= \begin{cases}
			\left(\frac{1}{6} d_{1}^3 + \frac{1}{2} (1 - 2 c_1 - e^{-c_1})d_{1}^2 - (1 - c_1 - \frac{1}{2} c_1^2 - e^{-c_1}) d_1\right) e^{-d_3} & T > \overlap + 3 \dead\\
				 1 - e^{-d_1} - d_1 e^{-d_2} - \frac{1}{2} (d_2^2 - c_1^2) e^{-d_3}       & T \in [\overlap + 2 \dead, \overlap + 3 \dead]\\
				 0                                               & \text{else}
			 \end{cases}\nonumber\\
	  &= \begin{cases}
			\bigg(\frac{1}{6} (d_{0}^3 - 3 c_1 d_0^2 + 3 c_1^2 d_0 - c_1^3) + \frac{1}{2} (1 - 2 c_1 - e^{-c_1})(d_{0}^2 - 2 c_1 d_0 + c_1^2) &\\
			  \quad \quad - (1 - c_1 - \frac{1}{2} c_1^2 - e^{-c_1}) (d_0 - c_1)\bigg) e^{-d_3} & T > \overlap + 3 \dead\\
				 1 - e^{-d_1} - (d_0 - c_1) e^{-d_2} - \frac{1}{2} ((d_0 - 2 c_1)^2 - c_1^2) e^{-d_3}       & T \in [\overlap + 2 \dead, \overlap + 3 \dead]\\
				 0                                               & \text{else}
			 \end{cases}\nonumber\\
\end{align}
\end{subequations}


We now present the guess for $n >= 1$:
\begin{subequations}
  \begin{align}
		p_{\rho, T, \dead}(n-1)
	  &= \begin{cases}
			\left(1 + \frac{n-1}{d_1}(1-e^{-\rho \dead})\right) \frac{1}{(n-1)!} d_{1}^{n-1} e^{-d_{n-1}} & T > \overlap + (n - 1) \dead\\
				 1 - \frac{e^{-d_1} \Gamma(n-1, d_1 e^{\rho\dead})}{(n-2)! e^{-d_1e^{\rho\dead}}}                       & T \in [\overlap + (n - 2) \dead, \overlap + (n - 1) \dead]\\
				 0                                               & \text{else}
		\end{cases} \label{eq:mod-Pois-guess}
	\end{align} 
	Note that in the limit $\dead \to 0$ we have $p_{\rho, T, \dead}(n-1) = \frac{1}{(n-1)!} (\rho T)^{n-1} e^{-\rho T}$,
	which was the (correct) guess the Poisson distribution.  Thus this guess reduces to the Poisson distribution as should be
	expected.

	Using Eq. (\ref{eq:mod-Pois-guess}), we can write $b_n(t)$, the probability that the first observation occurs at $t$ and $n - 1$
	observations occur during the reamining time:
\begin{align}
	b_n(t)
	  &= \begin{cases}
			p(t) p_{\rho, T - t - \dead, \dead}(n - 1) & t \in [0, T - \overlap - \dead]\\
				 0                                & \text{else}
			 \end{cases}\nonumber\\
	  &= \begin{cases}
			\rho e^{-\rho t} \left(\frac{1}{2} (d_{1} - \rho \dead - \rho t)^2 + (1-e^{-c_1}) (d_{1} - \rho \dead - \rho t) \right)e^{-d_2 + \rho \dead + \rho t} dt & T - \dead > \overlap + 2 \dead\\
			 & \text{and } t \in \left[0, \frac{d_1}{\rho}\right]\\
				 \rho e^{-\rho t} \left(1 - e^{-d_1 + \rho \dead + \rho t} - (d_1 - \rho \dead - \rho t) e^{-d_2 + \rho \dead + \rho t}\right) dt & T - \dead \in [\overlap + \dead, \overlap + 2 \dead]\\
				 & \text{and } t \in \left[0, \frac{d_1}{\rho}\right]\\
				 0                                               & \text{else}
		\end{cases}\nonumber\\
	  &= \begin{cases}
			\rho \left(\frac{1}{2} (d_{2} - \rho t)^2 + (1-e^{-c_1}) (d_{2} - \rho t) \right)e^{-d_3} dt & T > \overlap + 3 \dead\\
			 & \text{and } t \in \left[0, \frac{d_1}{\rho}\right]\\
				 \rho \left(e^{-\rho t} - e^{-d_2} - (d_2 - \rho t) e^{-d_3}\right) dt & T \in [\overlap + 2 \dead, \overlap + 3 \dead]\\
			 & \text{and } t \in \left[0, \frac{d_1}{\rho}\right]\\
				 0                                               & \text{else}
		\end{cases}\label{eq:b_3-pre}
\end{align}
With the $u$ and $v$ substitutions, we get
\end{subequations}

For this document, we do not validate the dead-time modified Poisson distribution.
To validate, we'd compare to the distribution in the existing literature and
also obtain analytic forms for the moments.  We have done some of this work
and can show this distribution numerically has the expected centralized moments,
but futher analysis to look at the moments is time-consuming and beyond the
scope of the derivation we present in this document.

\end{document}
