\documentclass{article}

% Language setting
% Replace `english' with e.g. `spanish' to change the document language
\usepackage[english]{babel}

% Set page size and margins
% Replace `letterpaper' with `a4paper' for UK/EU standard size
\usepackage[letterpaper,top=2cm,bottom=2cm,left=3cm,right=3cm,marginparwidth=1.75cm]{geometry}

% Useful packages
\usepackage{afterpage}
\usepackage{amsmath}
\usepackage{graphicx}
\usepackage[colorlinks=true, allcolors=blue]{hyperref}
\usepackage{subcaption}
\newcommand{\dead}{\tau_{\text{dead}}}
\newcommand{\overlap}{t_{\text{over}}}


\begin{document}

{\Large \bf Statistical Inference in the Presence of Dead-Time}\\ \\
\smallskip
{\bf Background:}  When models involve stochastics, statisticians distinguish between ``forward'' models, that describe
measurable parameters in terms of underlying model parameters, and ``inverse models'', that describe underlying model
parameters in terms of measured parameters.  Statisticians treat the outcomes of these two approaches separately, labelling the
former as predictions and the latter as inferences.  If a forward model is known, there are standard Bayesian
techniques that can be use to obtain an inverse model so that inferences can be made from measurements.

\noindent {\bf Problems:}  The typical model of dead-time, a period of time during which detectors cannot receive additional signal,
used within ISR is a forward model.
In several projects, this
forward model is being used as an inverse model without alteration to make inferences about physical parameters.
A review of the literature could not find a discussion of the proper statistical inversion process of the dead-time model
nor the proper ``subtraction'' of noise from the measurement.

The typical dead-time model in fact describes the relationship between the expectation of the forward model, and the
proper inversion of this model requires a description of the full ``forward'' distribution.
A description of the full forward distribution is within the literature; however, we
note that this description has some issues with numerical stability and
analytic tractability.

\noindent {\bf Specific Aims:}
In this proposal, we propose
\begin{enumerate}
	{\item{{\bf Derivation of the Dead-time Modified Poisson Distribution:}  We are partly through a derivation that generalizes
		the Poisson Distribution in the presence of dead-time and propose to complete this work.
		We are confident that the results of this derivation
		will resolve the numerical stability and analytic tractability issues with the current distribution described
		within the literature.}}
	{\item{{\bf The development of a methodology detailing statistical inference in the presence of dead-time:} We propose to invert the Dead-time Modified Poisson Distribution
		using Baye's theorem.  Once the Dead-time Modified Poisson Distribution is ready, we
		can introduce reasonable parameterizations of prior distributions that both should capture our expectation
		of noise on our detected signal as well as our expectation of the frequency of our signal.}}
\end{enumerate}

\noindent{\bf Intellectual Merit:}
The above works should produce two or more articles fit to be considered for publication.  The first will be aimed toward
a theoretical statistical journal and the second towards a nuclear instrumentation journal.  As we could not
readily find the appropriate statistical inversion process within the scientific literature, the addition of these two
publications should improve the statistical methodology employed by scientists when they are making inferences
from detectors where dead-time effects are important.  Specifically, this work will describe the correct procedure
by which noise can be accounted for within statistical inferences from measurements where dead-time is relevant and
it will provide both expectations for model parameters and their uncertainty from these measurements.

\noindent {\bf Broader Impacts:}
Part of this work will be done by the junior scientist, Karl Schmidt, who is currently working on a Masters of Science in Data Science at ...
This experience will provide Karl with further training within this discpline and further mentorship opportunities
with the senior scientists on the proposal, Drs. Predrag Punosevac and Brandon Zerbe, and it will provide
him with some publications that he can used to communicate his burgeoning expertise.

Also, the results of this work will be communicated with scientists throughout ISR.  The proper implementation
of statistical inference in the presence of dead-time should both provide a more nuanced statistical description
of the inferred parameters as well as improve the confidence ISR scientists have within determination of physical 
parameters made from measurements where dead-time effects are present. 

\end{document}

which is 
creating correct dInversion of the dead-time model is often incorrectly done within detector literature
as well as some ISR projects,
and proper inversion of this model would (1) improve our estimates of physical parameters made from detector 

\section{Introduction}

Many sensors operate by detecting discreet events
such as the arrival of some real or virtual particle
that is then used to infer information about some remote event.
The first step in this inference typically requires relating the measured number of arrivals, $n$,
back to a rate parameter, $\rho$.  Physical theory and experimental data on the emission and transmission
processes describes the
inference procedure related $\rho$ to parameters of the remote event.
%Namely, theory and data describe at what rate
%such particles are emitted from an event and then also inform how these rates evolve as they are transmitted
%from the event to the detector; ``reversing'' this knowledge allows us to infer details about the
%remote event from the inferred rate deterimined in the first step.
%Therefore, to be able to most accurately and precisely
%infer the parameters relevant to a remote event from a single or series of local measurements requires
%an understanding of the fundamental physics involved in the emission, transmission, and detection processes.
%
Within this document, we focus on models for the physics of detection and specifically
the inference of $\rho$ from an observation of $n$ arrivals.  We note that we do not treat
the physical observation that an incoming particle only has some probability of interacting
with the detector even when
it passes ``through'' it; we assume that such effects are incorporated within the inference process elsewhere.

\subsection{Background}

One of the largest effects in the physics of detection is randomness, and
there is a long history of 
models exploring the effect of randomness on the arrival of events when the
average rate of arrival is given by the rate parameter, $\rho$.
Through modelling inter-particle arrival distribution times by the exponential distribution,
now known as a Poisson process model, Poisson was able to
derive an expression for the probability of $n$ arrivals being observerd
during a fixed period of $T$ for any value of $n$;
this collection of probabilities is now known as the Poisson distribution.
%For conciseness, we call the probability for a specific $n$ within the Poisson distribution
%``the probability of $n$ given $\rho$'' as the probability of $n$ implicitly depends on the
%value of $\rho$.
The Poisson distribution is often used to understand the statistics of detection;
however, further considerations needs to be included in the modelling before analogous distributions may be
used for inference with modern detectors.

\subsubsection{Dead-time}

The first effect is physical and is called the dead-time effect. 
Dead-time is a period of time, mathematically denoted by $\dead$, after which
an arrival is detected and during which further arrivals cannot be detected.  Dead-time
arises physically from quantum characteristics of the materials/chemicals
that are utilized to initiate the signal from an incoming event for down-stream processing within the detector.
In this document, we are concerned with what is technically called non-paralyzable dead-time,
where $\dead$ is treated as a constant; however, we shorten this to just dead-time
and warn the reader to keep this in mind when reading other literature.

The dead-time effect has been treated within
the literature to varying degrees.  Denoting the number of detected events by $n$ and the period over which
detection occurs as $T$,
the typical ``dead-time model'' relates
the expectation of the number of detected events, $<n>$, to the other parameters by
$<n> = \frac{\rho T}{1 + \rho \dead}$.

When presented in most textbooks, though, the
expectation notation is often dropped in favor of using $n$.
Without the indication that we are discussing a distribution, though, many
scientists invert this equation to obtain the incorrect equation $\rho = \frac{n}{T - n \dead}$ [refs];
however, this is not proper way to invert this relations.  The proper way to invert such distribution
relations is known as statisical inference, and such methodologies require a description of the full distribution.

Work done in the 1960-70's [ref] did an analysis of ``renewal processes''
in order to obtain an approximate expression for not only the average but
the full probability distribution.  However, the expression for this distribution is written
in terms of differences of Gamma functions that has some numerical stability issues when the Gamma functions
are close to one another.
We are currently part-way through a generalization of the derivation of the Poisson distribution in the presence
of dead-time for which we expect to obtain an easier to work-with Dead-time modified Poisson Distribution, for which
we are asking for funds to finish.  We expect that the resulting derivation will be appropriate for
publication in a peer-reviewed journal in theoretical statistics.

Once we have an analytically tractable distribution, we propose doing statistical inference
through Bayesian inference.  This will allow us to not only obtain the statistically correct
inversion of the typical dead-time model, but it will also allow us to appropriately treat
sources of noise that may be arriving in addition to the rate parameter of interest.

%
%There is a way to transform a forward model to an inverse model using
%statistical techniques; to understand what these techniques do, we first need to
%elucidate an aspect of the Poisson distribution.
%To do so,
%we reframe the ``inversion'' process by asking the following question:
%``What is the best estimate of $\rho$ given we made the measurement $n$?''
%Implicit in this question is an inversion of implicit conditional relation 
%between $\rho$ and $n$ --- that is, the Poisson distribution 
%describes probabilities of $n$ conditioned on the value of $\rho$ whereas we are interested in the probabilities 
%where this conditional statement is inverted, i.e. probabilites of
%$\rho$ conditioned on the value of $n$.
%We call the ``inverted conditional'' distribution of the dead-time modified Poisson distribution
%the inverted dead-time distribution.  

%In statistical theory, a distribution derived from a forward model can be
%related to the distribution derived from an inverse model through
%the application of standard Bayesian techniques.  Specically,
%the ``forward'' and ``inverse'' probabilities are related through
%Bayes' theorem, so knowing one, as we do here, gives most of the insight
%we need to be able to derive the other.
%On the other hand, Bayes' theorem introduces a second probability distribution typically
%called the prior probability, which is usually ``known'' through modelling.
%The prior probability modelling
%introduces additional parameters that are then typically trained against
%data as well as theoretical expectation.  
%Usually, a portion or all of the training data is taken from signal originating from background events
%from which the designers of the detector wish to discriminate signals originating from events
%of interest, and in this way, background is automatically incorporated into the underlying inference
%distribution without further work.

\subsection{Prospective}

%In this document, we use Bayesian techniques to derive
%the analytic ``inverted dead-time distribution''
%from the analytic dead-time modified
%Poisson distribution
%with the goal of  the statistically correct inference
%of both the rate and it's uncertainty.
%The structure of the remainder of this document follows:
\begin{enumerate}
	{\item{{\bf Section \ref{sec:dead-time-dist}: The dead-time modified Poisson distribution} -- we present the generalization
		of the Poisson distribution in the presence of dead-time that resolves the issues with the previous
		expression of this distribution, and we very briefly overview this derivation.}}
	{\item{{\bf Section \ref{sec:rate-dist}: The  inverted dead-time distribution} -- we derive
		the  inverted dead-time distribution.  This derivation is short but requires the development of
		a model of the prior probability, which we describe in detail.}}
	{\item{{\bf Section \ref{sec:discussion} Discussion}:  We tabulate all parameters in their model and describe how they
		can be estimated from data.
		To provide additional practical context for the inference, 
		we present visualizations for a specific choice of the values of the
		parameters and walk through how the estimates of the rate and it's uncertainty are
		obtained from the resulting inverted dead-time distribution after measurements are made.}}
\end{enumerate}

%We
%denote this distribution by $p_{T, \dead}(n | \rho)$ where we have $T$ and $\dead$
%in the sub-script to indicate that they are fixed constants and $n | \rho$ within the function
%is standard nomenclature for conditional probabilities; within this notation,
%the Poisson distribution is given by $p_T(n | \rho)$.
%$p_{T, \dead}(\rho | n)$;

%Adopting the parameterizatiSpeicifically, textbooks
%quantify the dead-time model as $n = \frac{\rho T}{1 + \rho \tau}$

%In statistics, a conditional distribution and its ``inverse'' are related through Bayes' theorem which
%is given by $p(\rho | n) = \frac{p(n | \rho) p(\rho)}{p(n)}$ and is read as ``the inverted
%dead-time probability is
%$n$ is the probability of $n$ given $\rho$ times the probability of $\rho$ divided by the probability of $n$.
%In Bayesian statistics, the probability of $\rho$ is known as the ``Prior probability'', and if it is correctly
%modelled and the probability of $n$ given $\rho$ is known, the 
%Here and since we 
%have an expression for the dead-time modified Poisson distribution, we have an
%analytic pathway to obtain the expression for the inverted dead-time distribution that not only treats
%the ``inversion'' process appropriately but that can also incorporate background noise with statistical rigour.

%\section{The dead-time modified Poisson distribution}\label{sec:dead-time-dist}

%\section{The  inverted dead-time distribution}\label{sec:rate-dist}

%\section{Discussion}\label{sec:discussion}

\end{document}
